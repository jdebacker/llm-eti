\begin{table}[!htbp] \centering
  \caption{Simulated Elasticity of Taxable Income by Model}
\begin{tabular}{lcc}
\\[-1.8ex]\hline
\hline \\[-1.8ex]
 & GPT-4o & GPT-4o-mini \\
\hline \\[-1.8ex]
\multicolumn{3}{l}{\textbf{Panel A: Summary Statistics}} \\
\\[-1.8ex]
Mean ETI & $0.364$ & $1.280$ \\
Median ETI & $-0.000$ & $1.140$ \\
Share with no response & 80.5% & 3.1% \\
Standard deviation & 2.534 & 0.756 \\
25th percentile ETI & $0.000$ & $0.833$ \\
75th percentile ETI & $0.000$ & $1.667$ \\
Number of responses & 15,985 & 15,999 \\
\\[-1.8ex]
\multicolumn{3}{l}{\textbf{Panel B: Regression Coefficients}} \\
\\[-1.8ex]
Constant & $0.440^{***}$ \\ (0.056) & $1.224^{***}$ \\ (0.017) \\
Income (\$100k) & $-0.061^{*}$ \\ (0.034) & $0.045^{***}$ \\ (0.014) \\
MTR Change & $-6.797^{***}$ \\ (0.941) & $0.671^{***}$ \\ (0.210) \\
Income $\times$ MTR & $1.123^{**}$ \\ (0.569) & $0.934^{***}$ \\ (0.168) \\
$R^2$ & 0.020 & 0.028 \\
\hline
\hline \\[-1.8ex]
\multicolumn{3}{p{0.95\linewidth}}{\textit{Notes:} 
Heteroskedasticity-robust standard errors in parentheses. 
The empirical literature typically finds average elasticities of taxable income (ETI) around 0.2-0.3, 
with substantial shares of taxpayers not responding to tax changes due to optimization frictions. 
GPT-4o's predictions align more closely with these empirical patterns, generating a mean ETI of 0.36 
compared to 1.28 from GPT-4o-mini, and showing more non-response 
(81% vs. 3%). 
Panel B reports coefficients from regressions of ETI on income (in \$100k), marginal tax rate changes (in percentage points), 
and their interaction. The low $R^2$ values indicate substantial heterogeneity in responses beyond these systematic patterns.} \\
\multicolumn{3}{r}{$^{***}p<0.01$, $^{**}p<0.05$, $^{*}p<0.1$} \\
\end{tabular}
\end{table}