\section{Results}

\subsection{Model Comparison}
Our analysis reveals substantial differences in how the two models simulate taxpayer responses to marginal tax rate changes. Table~\ref{tab:regression} presents the main results, comparing response patterns across models. The more sophisticated model (\modelname{GPT-4o}) generates a mean \eti{} of 0.364, remarkably close to empirical estimates of 0.2-0.3 from the literature. In contrast, \modelname{GPT-4o-mini} produces a substantially higher mean \eti{} of 1.280, suggesting excessive responsiveness to tax changes.

\input{tables/regression_results}

Figure~\ref{fig:eti_dist} illustrates the stark differences in \eti{} distributions between models. \modelname{GPT-4o}'s distribution shows a pronounced mass point at zero, reflecting realistic optimization frictions, with a long right tail similar to patterns observed in empirical studies. \modelname{GPT-4o-mini}'s distribution is more symmetric and shifted rightward, suggesting systematic over-prediction of behavioral responses.

\input{figures/eti_distribution}

\subsection{Response Heterogeneity}
The regression results in Panel B of Table~\ref{tab:regression} reveal interesting patterns in response heterogeneity across income levels and tax rate changes. Several key findings emerge:

\begin{enumerate}
    \item \textbf{Income Gradient}: \modelname{GPT-4o} exhibits a slightly negative income gradient ($-0.061$ per \$100,000), while \modelname{GPT-4o-mini} shows a positive gradient ($0.045$ per \$100,000). Figure~\ref{fig:income_gradient} illustrates this contrast.
    
    \item \textbf{Tax Rate Sensitivity}: The models show markedly different sensitivities to \mtr{} changes. \modelname{GPT-4o} demonstrates a stronger response ($-6.797$) compared to \modelname{GPT-4o-mini} ($0.671$), as shown in Figure~\ref{fig:tax_response}.
    
    \item \textbf{Interaction Effects}: Both models show significant interaction effects between income and tax rate changes, though with different magnitudes (\modelname{GPT-4o}: 1.123, \modelname{GPT-4o-mini}: 0.934).
\end{enumerate}

\input{figures/income_gradient}
\input{figures/tax_response}

\subsection{Non-Response Patterns}
Perhaps the most striking difference between the models lies in their non-response patterns. As shown in Figure~\ref{fig:response_rate}, \modelname{GPT-4o} exhibits no response in 80.5\% of cases, closely matching empirical evidence of substantial optimization frictions. In contrast, \modelname{GPT-4o-mini} shows an implausibly high response rate, with only 3.1\% of cases showing no behavioral change.

\input{figures/response_patterns}

This pattern of non-response in \modelname{GPT-4o} varies systematically with both income levels and the magnitude of tax changes, as documented in Table~\ref{tab:summary_stats}. The presence of substantial non-response, particularly for smaller tax changes, aligns well with the empirical literature on optimization frictions in taxpayer behavior.

\input{tables/summary_stats}

\subsection{Robustness and Additional Analyses}
We conduct several additional analyses to verify the robustness of our findings:

\begin{enumerate}
    \item \textbf{Directional Asymmetry}: Figure~\ref{fig:tax_direction} shows response patterns separately for tax increases and decreases, revealing similar patterns across both models.
    
    \item \textbf{Income Range Effects}: Our results remain robust across different income ranges within our sample (\$50,000-\$200,000), though the precision of estimates naturally decreases at the extremes of the distribution.
    
    \item \textbf{Alternative Specifications}: The low $R^2$ values in our main specifications (0.020 for \modelname{GPT-4o} and 0.028 for \modelname{GPT-4o-mini}) persist across alternative functional forms, suggesting that the heterogeneity in responses is a robust feature of the data rather than a artifact of our specification choices.
\end{enumerate}

\input{figures/tax_direction}

These results collectively suggest that model sophistication substantially influences the quality of simulated economic behavior, with the more advanced model generating patterns that more closely match empirical evidence from the tax literature.